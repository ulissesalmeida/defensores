\chapter{Os Atributos}
\label{ch:attributes}

Os atributos são extremamente importantes pois determinam o que o personagem é capaz de fazer. Cada um deles tem sua importância e são vitais na estratégia de criação do personagem. Os atributos influenciam os atributos secundários e estão diretamente ligados ao poder de combate. 

\section{Força}

Pontos gastos em Força(F) representa que o personagem tem treinamento em combate corpo-a-corpo, ou seja, para combates de curta distância é sua principal fonte de dano, manobras e ainda determina a quantidade de kilos que ele é capaz de mover.

\begin{description}
\item[FA] Para combates corpo-a-corpo a Força é a principal contribuídora para para sua Força de Ataque, onde sua FA final será: (FX2) + Habilidade. Personagens sem pontos em força e quer lutar corpo-a-corpo, recebe -4 em sua FA. Personagem com F3 e H4, a FA final será 10(3*2 + 4). 
\item[Peso] Para cada ponto em Força, seu personagem é capaz de carregar 100kg. Personagens com F0, movem 50kg. Para mover algo no limite do peso da sua força não é necessário testes, para mover coisas além da sua força, sua velociade é reduzia a 1/4 e é necessário passar em teste de dificuldade (3 + Pontos de Força necessário). Exemplo: Para mover algo de 700kg, é necessário passar em um teste de dificuldade 10 (3 + 7).
\item[Manobras] Há várias manobras e vantagens especiais exclusivas para personagens que lutam corpo-a-corpo, veja mais no capítulo Regras de combate(pag. \pageref{ch:combat})
\end{description}

\fbox{No sistema 3D\&T o personagem era capaz de carregar muito mais peso com seus pontos em força, porém era difícil de controlar e lembrar. Com 100kg por força, o nível de poder é menor e fácil de lembrar, pois basta multiplicar o valor da força por 100kg. F3 é 300kg a capacidade, F8 é 800kg, no sistema antigo quanto era capacidade para F12 mesmo? Tem que consultar o manual, desse jeito dispensa o manual e os personagens são mais fracos.}

\section{Habilidade}

Habilidade(H) é a destreza e agilidade do seu personagem, esse atributo ajuda no combate em todos aspectos além um dos atributos principais para uso de perícias. Todo teste que exige esquiva, movimento rápido, acrobacias, saltos e etc, esse é o atributo chave.

\begin{description}
\item[FA] Tanto para combates corpo-a-corpo e combate a distância, a Habilidade contribui! onde sua FA final será: (FX2) + Habilidade. Personagens sem pontos em força e quer lutar corpo-a-corpo, recebe -4 em sua FA. Personagem com F3 e H4, a FA final será 10(3*2 + 4). 
\item[FD] Tanto para combates corpo-a-corpo e combate a distância, a Habilidade contribui! onde sua FA final será: (FX2) + Habilidade. Personagens sem pontos em força e quer lutar corpo-a-corpo, recebe -4 em sua FA. Personagem com F3 e H4, a FA final será 10(3*2 + 4). 
\item[Velocidade] Para cada ponto em Força, seu personagem é capaz de carregar 100kg. Personagens com F0, movem 50kg. Para mover algo no limite do peso da sua força não é necessário testes, para mover coisas além da sua força, sua velociade é reduzia a 1/4 e é necessário passar em teste de dificuldade (3 + Pontos de Força necessário). Exemplo: Para mover algo de 700kg, é necessário passar em um teste de dificuldade 10 (3 + 7).
\item[Salto] Há várias manobras e vantagens especiais exclusivas para personagens que lutam corpo-a-corpo, veja mais no capítulo Regras de combate(pag. \pageref{ch:combat})
\item[Perícias] Há várias manobras e vantagens especiais exclusivas para personagens que lutam corpo-a-corpo, veja mais no capítulo Regras de combate(pag. \pageref{ch:combat})
\item[Testes] Há várias manobras e vantagens especiais exclusivas para personagens que lutam corpo-a-corpo, veja mais no capítulo Regras de combate(pag. \pageref{ch:combat})
\item[Manobras] Há várias manobras e vantagens especiais exclusivas para personagens que lutam corpo-a-corpo, veja mais no capítulo Regras de combate(pag. \pageref{ch:combat})
\end{description}

\section{Resistência}

\section{Mente}

\section{Armadura}

\section{Poder de Fogo}

\section{Atributos Secundários}

\subsection{Pontos de Vida}

Pontos de Vida(PVs) representam a vitalidade do personagem, quando seu personagem atinge 0PVs fica incapaz de lutar ou realizar alguma ação importante.

\subsection{Pontos de Magia}

Pontos de Magia(PMs) representam a energia mágica, espiritual, física, ki...etc do personagem, quando seu personagem atinge 0PMs fica incapaz de usar habilidades especiais.

\subsection{Pontos de Ação}

Pontos de Ação (PAs) representam a sorte, a força de vontade sobrenatural ou qualquer coisa que permita seu personagem fazer do impossível, possível.

\subsection{Força de Ataque}

\subsection{Força de Defesa}

\subsection{Fadiga}

\subsection{Velocidade}

\subsection{Alcance de Poder de Fogo}

\subsection{Tipo de Dano Especializado}
